\documentclass{article}

\usepackage{amsmath}
\usepackage{fancyhdr}
\usepackage{graphicx}
\graphicspath{{}}

%% some colours
\usepackage{color}
\definecolor{deepblue}{rgb}{0,0,0.5}
\definecolor{deepred}{rgb}{0.6,0,0}
\definecolor{deepgreen}{rgb}{0,0.5,0}
\definecolor{backcolour}{rgb}{0.95,0.96,0.93}

%%%%%%%%%%%%%% CODE STUFF %%%%%%%%%%%%%%
%%%%%%%%%%%%%%%%%%%%%%%%%%%%%%%%%%%%%%%%
\usepackage{cprotect} % to be used in sol
\usepackage{listings} % for code display
% setting code style
\newcommand\pythonstyle{\lstset{
        language=Python,
        backgroundcolor=\color{backcolour},
		basicstyle=\footnotesize,
		otherkeywords={self},
		keywordstyle=\footnotesize\color{deepblue},
		emph={__init__},
		emphstyle=\footnotesize\color{deepred},
		stringstyle=\color{deepgreen},
		frame=single,
		showstringspaces=false  ,
		breaklines=true,
		numbers=left,
		numberstyle=\footnotesize,
		tabsize=4,
		breakatwhitespace=false
	}}

% Python environment
\lstnewenvironment{python}[1][]{
    \pythonstyle
    \lstset{#1}
}{}

% Python for external files
\newcommand\pythonexternal[2][]{{
    \pythonstyle
    \lstinputlisting[#1]{#2}
}}

% Python for inline
\newcommand\pythoninline[1]{{\pythonstyle\lstinline!#1!}}

%%%%%%%%%%%%%%%%%%%%%%%%%%%%%%%%%%%%%%%%
% setting the style for ex documents
\pagestyle{fancy}
\fancyhf{}
\fancyhead[L]{\thetitle}
\fancyhead[C]{}
\fancyhead[R]{\theauthor}
\renewcommand{\headrulewidth}{0.4pt} %obere Trennlinie
\fancyfoot[L]{Due: \thedate}
\fancyfoot[R]{\thepage} %Seitennummer
\renewcommand{\footrulewidth}{0.4pt}

% include solutions
\newcommand\sol[1]{{\large\textbf{\\Solution:}}#1}
\usepackage{tikz}
\usetikzlibrary{arrows,automata}

\title{BPP Exercise 6 - Sorting and IO}
\author{A. Hain, M. Nipshagen}
\date{14.05.2018, 08:00}

\makeatletter
\let\thetitle\@title
\let\theauthor\@author
\let\thedate\@date
\makeatother

% do not include solutions
% \renewcommand\sol[1]{}


\begin{document}

The deadline for this exercise sheet is \textbf{Monday, \thedate.}
%
%\section*{Introductory Words}
%In case we have some information that doesn't directly concern the current exercises.
%

\section{Warm-Up: Debug Like There's No Tomorrow}

This week is going to be all about fixing errors in the code instead of
building your own programs from scratch.\\
Attached to this sheet there is a folder \texttt{debug\_me} full of small
Python scripts. All of them throw some sort of exception.\\
Fix the Python files to get rid of the exceptions and to make them do what
they were supposed to do. Run your version of the code to make sure it
actually works.
\textbf{Then put a small comment in each file explaining what was the problem.}\\

\section{Tic-tac-toe}

Tic-tac-toe is a classic game easily played on paper with a pencil, in
which both players, \textit{X} and \textit{O}, try to win by getting three
of their symbols in one row, column or diagonal of the 3x3 grid.

\begin{figure}[h]
\centering
\includegraphics[width=0.4\textwidth]{tictactoe}
\label{fig:Tic-tac-toe}
\caption{Example Tic-tac-toe game}
\small{Taken from \url{https://commons.wikimedia.org/wiki/File:Tic_tac_toe.svg}}
\end{figure}

\noindent Attached you will find a file named \texttt{tictactoe.py}.\\
This file contains the game, adaptable even to grids of other sizes.
Unfortunately, it's not working, but keeps throwing exceptions.\\
Fix the exceptions in the code by running the program and getting familiar
with the code. Similar to the first exercise, put a comment into the file
and \textbf{explain what you changed}.\\

\noindent\textit{Hint:} All of the errors can be fixed by either modifying the same
line or adding one other. Before you hand in the homework, play the game
for a little to actually make sure it is not crashing.\\
\textbf{\textit{Bonus:}} As soon as the game does not crash anymore, you might
notice that even though it's now technically working, it is not working
\textit{properly} and therefore not completely doing what is is supposed to
do. There's two reasons for that - find and fix them as well.


\end{document}
