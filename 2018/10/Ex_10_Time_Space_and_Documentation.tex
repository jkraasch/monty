\documentclass{article}

\usepackage{amsmath}
\usepackage{fancyhdr}
\usepackage{graphicx}
\graphicspath{{}}

%% some colours
\usepackage{color}
\definecolor{deepblue}{rgb}{0,0,0.5}
\definecolor{deepred}{rgb}{0.6,0,0}
\definecolor{deepgreen}{rgb}{0,0.5,0}
\definecolor{backcolour}{rgb}{0.95,0.96,0.93}

%%%%%%%%%%%%%% CODE STUFF %%%%%%%%%%%%%%
%%%%%%%%%%%%%%%%%%%%%%%%%%%%%%%%%%%%%%%%
\usepackage{cprotect} % to be used in sol
\usepackage{listings} % for code display
% setting code style
\newcommand\pythonstyle{\lstset{
        language=Python,
        backgroundcolor=\color{backcolour},
		basicstyle=\footnotesize,
		otherkeywords={self},
		keywordstyle=\footnotesize\color{deepblue},
		emph={__init__},
		emphstyle=\footnotesize\color{deepred},
		stringstyle=\color{deepgreen},
		frame=single,
		showstringspaces=false  ,
		breaklines=true,
		numbers=left,
		numberstyle=\footnotesize,
		tabsize=4,
		breakatwhitespace=false
	}}

% Python environment
\lstnewenvironment{python}[1][]{
    \pythonstyle
    \lstset{#1}
}{}

% Python for external files
\newcommand\pythonexternal[2][]{{
    \pythonstyle
    \lstinputlisting[#1]{#2}
}}

% Python for inline
\newcommand\pythoninline[1]{{\pythonstyle\lstinline!#1!}}

%%%%%%%%%%%%%%%%%%%%%%%%%%%%%%%%%%%%%%%%
% setting the style for ex documents
\pagestyle{fancy}
\fancyhf{}
\fancyhead[L]{\thetitle}
\fancyhead[C]{}
\fancyhead[R]{\theauthor}
\renewcommand{\headrulewidth}{0.4pt} %obere Trennlinie
\fancyfoot[L]{Due: \thedate}
\fancyfoot[R]{\thepage} %Seitennummer
\renewcommand{\footrulewidth}{0.4pt}

% include solutions
\newcommand\sol[1]{{\large\textbf{\\Solution:}}#1}
\usepackage{setspace}

\title{BPP Exercise 10 - Time, Space and Documentation}
\author{A. Hain, M. Nipshagen}
\date{11.06.2018, 12:00}


\makeatletter
\let\thetitle\@title
\let\theauthor\@author
\let\thedate\@date
\makeatother


\newcommand\itemsub[1]{
	\begin{itemize}
		\item #1
	\end{itemize}
}

% do not include solutions
% \renewcommand\sol[1]{}


\begin{document}

The deadline for this exercise sheet is \textbf{Monday, \thedate.}

\section*{Introductory Words}
Remember that you need proper documentation to pass
the homework. The documentation doesn't need to be \textit{perfect}, but
everything that needs a docstring, should have a docstring.

\section{Warm-Up: The Final Countdown}
Write a script that uses the \texttt{datetime} module to make a countdown from
10 seconds to 0 seconds in which each second the remaining time is printed.\\
\textit{Bonus: } Make the countdown more dramatic. We don't care how exactly,
you can get creative! (One idea would be to include sounds).


\section{BirthdayCalc}
Usually, when asked how old we are, we can only answer in years (if even).
Let's get some more information on how much time we've already spent on this
lovely planet.\\
Write a module named \texttt{birthday\_calc} using which one can create
\texttt{BirthdayCalc} objects that can return how much time has passed since
one's birthday in different units.\\
A BirthDayCalc object is initialized with a day, a month and a year and a day
(of the birthday) and store this as a \texttt{datetime} object.
There have to be following methods in the class:
\begin{itemize}
	\item \texttt{birthday\_calc}, which returns your birthday as a datetime object
	\item \texttt{years\_since\_birth},
	\item \texttt{months\_since\_birth},
	\item \texttt{days\_since\_birth},
	\item \texttt{hours\_since\_birth},
	\item \texttt{minutes\_since\_birth} and
	\item \texttt{seconds\_since\_birth}, each of which will return an \textbf{integer}
	containing how much time has passed since the birthday in the respective unit.
\end{itemize}

\noindent Attached you will find a module named \texttt{birthday\_fun}. This is a test
class which you can run to test your implementation.\\
\textit{Bonus: }Extend \texttt{BirthDayCalc} by a function that returns the days
left until the user's next birthday. Also extend \texttt{birthday\_fun} such
that the user is able to use this new function.
\textit{Note: }You do not need to take care of error handling this time. The test
class will do this for you.\\
\textit{Note 2: }Instead of looking out for leap years, you can assume that a
year has exactly 365.25 days.




\end{document}
