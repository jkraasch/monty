\documentclass{article}

\usepackage{amsmath}
\usepackage{fancyhdr}
\usepackage{graphicx}
\graphicspath{{}}

%% some colours
\usepackage{color}
\definecolor{deepblue}{rgb}{0,0,0.5}
\definecolor{deepred}{rgb}{0.6,0,0}
\definecolor{deepgreen}{rgb}{0,0.5,0}
\definecolor{backcolour}{rgb}{0.95,0.96,0.93}

%%%%%%%%%%%%%% CODE STUFF %%%%%%%%%%%%%%
%%%%%%%%%%%%%%%%%%%%%%%%%%%%%%%%%%%%%%%%
\usepackage{cprotect} % to be used in sol
\usepackage{listings} % for code display
% setting code style
\newcommand\pythonstyle{\lstset{
        language=Python,
        backgroundcolor=\color{backcolour},
		basicstyle=\footnotesize,
		otherkeywords={self},
		keywordstyle=\footnotesize\color{deepblue},
		emph={__init__},
		emphstyle=\footnotesize\color{deepred},
		stringstyle=\color{deepgreen},
		frame=single,
		showstringspaces=false  ,
		breaklines=true,
		numbers=left,
		numberstyle=\footnotesize,
		tabsize=4,
		breakatwhitespace=false
	}}

% Python environment
\lstnewenvironment{python}[1][]{
    \pythonstyle
    \lstset{#1}
}{}

% Python for external files
\newcommand\pythonexternal[2][]{{
    \pythonstyle
    \lstinputlisting[#1]{#2}
}}

% Python for inline
\newcommand\pythoninline[1]{{\pythonstyle\lstinline!#1!}}

%%%%%%%%%%%%%%%%%%%%%%%%%%%%%%%%%%%%%%%%
% setting the style for ex documents
\pagestyle{fancy}
\fancyhf{}
\fancyhead[L]{\thetitle}
\fancyhead[C]{}
\fancyhead[R]{\theauthor}
\renewcommand{\headrulewidth}{0.4pt} %obere Trennlinie
\fancyfoot[L]{Due: \thedate}
\fancyfoot[R]{\thepage} %Seitennummer
\renewcommand{\footrulewidth}{0.4pt}

% include solutions
\newcommand\sol[1]{{\large\textbf{\\Solution:}}#1}

\title{BPP Exercise 8 - 4P: Hints \& Help}
\author{A. Hain, M. Nipshagen}
\date{29.05.2018, 14:00}

\makeatletter
\let\thetitle\@title
\let\theauthor\@author
\let\thedate\@date
\makeatother

\newcommand\itemsub[1]{
	\begin{itemize}
		\item #1
	\end{itemize}
}

\setcounter{secnumdepth}{0}

\begin{document}
The deadline for this exercise sheet is \textbf{Tuesday, \thedate.}
\tableofcontents
\vspace{12pt}\noindent
\textbf{DISCLAIMER:} These are all just suggestions and not necessarily a complete
or the best approach to a solution. It just offers hints, general approaches
and ideas.
\pagebreak

\section{The Pseudocode}
The following pseudocode would be a possible (and simplified) way of doing it:
\begin{verbatim}
Let User Choose A Maze
Load Maze From File
Convert Maze Into List-like Structure
Print Maze Without Path
Find Start Point
Find Goal Point
Start Recursive Search:
    If Position Is Goal Positon:
        Done
    If Position Is Wall:
        Go Back
    If Position Is Already Visited:
        Go Back
    For All Directions:
        Search(NewPos)

If Way Found:
    Print Maze With Path
Else:
    Print Sad Message Since One Is Forever Trapped
If Save:
    Save Solved Maze To New File
\end{verbatim}
\pagebreak

\section{General Hints That Can Be More Or Less Helpful}
\begin{itemize}
    \item You need to check the 4 neighbouring fields of each cell
        \itemsub{be careful with the bounds of the grid}
    \item Each cell you visit needs to be checked before continuing to the neighbours
    \item Make sure to mark visited cells with a marker so you don't visit a cell twice
    \item The necessary checks are whether it is a wall, path, start, or goal cell
        \itemsub{make sure to return according results}
    \item Just like last week, part your code into smaller specific functions
        \itemsub{Like one function for reading mazes from files, one for converting them
        into a grid, and one for solving the maze then, and so on...}
\end{itemize}
\pagebreak

\section{Module Separation}
We suggest the modules:
\begin{itemize}
	\item \texttt{solver.py}:
	      This is where the actual recursion and solving happens.
	      Most importantly it implements the \texttt{solve} function, which
	      starts the search for a way
	\item \texttt{fileIO.py}:
	      This module is responsible for reading mazes from files and rebuild them
	      into a usable Python structure. It could also offer the possibility to write
          back to files, to make it easy to save solutions to mazes or save mazes
          inputted by the user
	\item \texttt{userIO.py}:
	      This module offers functions to display the maze in a nice way on the
	      terminal and interact with the user, like getting input, varifying the input, etc.
	\item \texttt{maze.py}:
	      This is what brings it all together. It is the "main" module of the whole pack.
	      It imports the functionality of the other 3 modules and runs the main program to
	      choose a maze, load the maze, solve the maze and print the solution
\end{itemize}
\pagebreak

\section{Suggested Functions In Module 1}
Module \texttt{maze.py} could contain:
\begin{itemize}
	\item \pythoninline{main():} the function that starts and runs it all. That's it.
\end{itemize}
\pagebreak

\section{Suggested Functions In Module 2}
Module \texttt{solver.py} could contain:
\begin{itemize}
	\item \pythoninline{WALL} a constant that tells the solver which character
	      represents a wall (you may default it to \pythoninline{#})
	\item \pythoninline{START} a constant that tells the solver which character
	      represents the start cell
	\item \pythoninline{GOAL} a constant that tells the solver which character
          represents the goal cell
    \item \pythoninline{PATH} a constant that tells the solver which character
        represents the visited cells - our taken path.
	\item \pythoninline{solve(maze, startpos, goalpos):} starts the search for
	      the path by calling the initial recursive search (where startpos,
	      and goalpos are tuples with coordinates)
	\item \pythoninline{solve_recursive(maze, currentpos, goalpos):} recursively
	      searches for a path through the maze by backtracking
	\item \pythoninline{findstart(maze):} returns the coordinates of the start
	      position
	\item \pythoninline{findgoal(maze):} returns the coordinates of the goal position
\end{itemize}
\pagebreak

\section{Suggested Functions In  Module 3}
Module \texttt{fileIO.py} could contain:
\begin{itemize}
    \item \pythoninline{MAZEFOLDER} a constant which represents the relative path to
        the folder in which the mazes are located
    \item \pythoninline{get_maze_list():} returns a list of filenames for
        available mazes. This relates to the \emph{Small-ish Bonus Task 1}
    \item \pythoninline{load_maze(filename):} returns the maze from the given file name
        inside the \pythoninline{MAZEFOLDER}
    \item \pythoninline{build_maze(raw_maze):} takes a maze read from a file and builds
        the maze's grid-like structure that we need to work with, then returns the grid
    \item \pythoninline{stringify_maze(grid):} converts a grid back into a maze that
        can be written into a file so it looks like the others
    \item \pythoninline{save_maze(maze, filename):} saves a maze into a file with filename in
        the \pythoninline{MAZEFOLDER}
    \item \pythoninline{save_solved(maze, filename):} saves a solved maze into a file with
        file name in the \pythoninline{MAZEFOLDER}, maybe with a special \texttt{solved} suffix
        This can of course use \pythoninline{save_maze} in the end so you don't duplicate code
    \item \pythoninline{get_maze_grid(filename):} consecutively calls \pythoninline{load_maze()}
        and \pythoninline{build_maze()} to return a grid. This is a convenience function
    \item \pythoninline{save_maze_grid(grid, filename, solved=True):} consecutively calls
        \pythoninline{stringify_maze()} and on of the save functions. Also a convenience function
\end{itemize}
\pagebreak

\section{Suggested Functions In  Module 4}
Module \texttt{userIO.py} could contain:
\begin{itemize}
	\item \pythoninline{show_welcome():} shows some introductory words, and maybe a
	      selection of available options to choose from
	\item \pythoninline{present_mazes(maze_list):} displays the mazes to choose from
	\item \pythoninline{get_user_choice(maze_list):} gets the choice of the user, and
	      checks if it is a valid choice, then returns the corresponding maze name
	\item \pythoninline{choose_maze(maze_list):} consecutively calls
	      \pythoninline{present_mazes} and \pythoninline{get_user_choice} to choose a maze
	      A convenience function
	\item \pythoninline{print_maze(maze):} prints the maze onto the terminal, however
	      fancy you would like to make it
	\item \pythoninline{read_maze():} reads in a maze from the user and returns it.
	      This is related to \emph{Small-ish Bonus Task 2} and might require some more
	      functionality / functions
\end{itemize}

\end{document}