\documentclass{article}

\usepackage{amsmath}
\usepackage{fancyhdr}
\usepackage{graphicx}
\graphicspath{{}}

%% some colours
\usepackage{color}
\definecolor{deepblue}{rgb}{0,0,0.5}
\definecolor{deepred}{rgb}{0.6,0,0}
\definecolor{deepgreen}{rgb}{0,0.5,0}
\definecolor{backcolour}{rgb}{0.95,0.96,0.93}

%%%%%%%%%%%%%% CODE STUFF %%%%%%%%%%%%%%
%%%%%%%%%%%%%%%%%%%%%%%%%%%%%%%%%%%%%%%%
\usepackage{cprotect} % to be used in sol
\usepackage{listings} % for code display
% setting code style
\newcommand\pythonstyle{\lstset{
        language=Python,
        backgroundcolor=\color{backcolour},
		basicstyle=\footnotesize,
		otherkeywords={self},
		keywordstyle=\footnotesize\color{deepblue},
		emph={__init__},
		emphstyle=\footnotesize\color{deepred},
		stringstyle=\color{deepgreen},
		frame=single,
		showstringspaces=false  ,
		breaklines=true,
		numbers=left,
		numberstyle=\footnotesize,
		tabsize=4,
		breakatwhitespace=false
	}}

% Python environment
\lstnewenvironment{python}[1][]{
    \pythonstyle
    \lstset{#1}
}{}

% Python for external files
\newcommand\pythonexternal[2][]{{
    \pythonstyle
    \lstinputlisting[#1]{#2}
}}

% Python for inline
\newcommand\pythoninline[1]{{\pythonstyle\lstinline!#1!}}

%%%%%%%%%%%%%%%%%%%%%%%%%%%%%%%%%%%%%%%%
% setting the style for ex documents
\pagestyle{fancy}
\fancyhf{}
\fancyhead[L]{\thetitle}
\fancyhead[C]{}
\fancyhead[R]{\theauthor}
\renewcommand{\headrulewidth}{0.4pt} %obere Trennlinie
\fancyfoot[L]{Due: \thedate}
\fancyfoot[R]{\thepage} %Seitennummer
\renewcommand{\footrulewidth}{0.4pt}

% include solutions
\newcommand\sol[1]{{\large\textbf{\\Solution:}}#1}
\usepackage{setspace}

\title{BPP Exercise 8 - 4P}
\author{A. Hain, M. Nipshagen}
\date{04.06.2018, 8:00}

\makeatletter
\let\thetitle\@title
\let\theauthor\@author
\let\thedate\@date
\makeatother


\newcommand\itemsub[1]{
	\begin{itemize}
		\item #1
	\end{itemize}
}

% do not include solutions
% \renewcommand\sol[1]{}


\begin{document}

The deadline for this exercise sheet is \textbf{Monday, \thedate.}

%\section*{Introductory Words}


\section{Warm-Up: The Land Before Time}
Many many years ago, our planet was the habitat of many different types of dinosaurs.\\
We modeled several kinds of them and their behavior (in a manner that may or may
not be a little simplified) in a file named \texttt{dinos.py}.\\
Look at the dino classes given in the file. State for each class the following:
\begin{itemize}
	\item{All superclasses}
	\item{All subclasses}
	\item{All classes it extends}
	\item{All functions it can use}
	\item{All functions it overrides}
\end{itemize}

\section{I want to ride my bicycle, I want to ride my bike}
Implement three classes: \textit{Bike}, \textit{Bicycle} and \textit{Motorbike}.\\
\textit{Bike} is the superclass of both \textit{Bicycle} and \textit{Motorbike}.\\
A \textit{Bike} has a number of seats and a number of gears. It can start being ridden and
end being ridden. Also, one is able to change the gears (to a gear number that
actually exists.)\\
A \textit{Bicycle}, additionally to the \textit{Bike}, has a bell that can be rung.\\
A \textit{Motorbike}, additionally to the \textit{Bike}, has a tank that can be filled.\\
Write a module of the classes according to the description and add another module
that tests them with some values. Don't forget to include proper documentation!




\end{document}
